%简单的介绍了Linux的基本使用,登陆,图形和文字的切换,账号之间的切换等
\section{linux基本使用}

\begin{frame}{Linux基础教程}{30分钟学会Linux}

\begin{center}
\includegraphics[width=0.5\textwidth]{images/basic/tux}
\end{center}
\end{frame}


\subsection{登录系统}


\begin{frame}{登录系统}
\begin{itemize}
\item login: 输入账户(比如root)
\item password: 输入密码,密码不回显,也不是显示{*}号
\item \# 表示超级用户(root)
\item \$ 表示普通用户
\item exit/CTRL+D 注销
\item poweroff/shutdown/init 0 关机
\end{itemize}

\end{frame}

\subsection{操作技巧}


\begin{frame}{操作技巧}
\begin{itemize}
\item 命令历史记录 history(/etc/profile)


\pause{\noindent }

\item 自动补全(TAB键)


\pause{}

\item 命令别名 alias (\textasciitilde{}/.bashrc \textasciitilde{}/.bash\_aliases)


\pause{}

\item 鼠标复制和粘贴(滚轮或者左右键)
\end{itemize}

\end{frame}

\subsection{在文件和目录中游走}


\begin{frame}{列出文件和目录}

\begin{tabular}{ll}
\hline 
ls & 列出文件和目录\tabularnewline
ls -a & 列出所有文件和目录(包括隐藏文件)\tabularnewline
mkdir \emph{dir} & 创建目录\tabularnewline
cd dir & 改变当前目录到dir\tabularnewline
cd \textasciitilde{} & 改变当前目录到用户主目录\tabularnewline
cd .. & 改变当前目录到上一级目录\tabularnewline
pwd & 显示当前目录\tabularnewline
\hline 
\end{tabular}


\end{frame}

\subsection{文件操作}


\begin{frame}{文件操作}

\begin{tabular}{ll}
\hline 
cp file1 file2 & 拷贝文件file1,命名为file2\tabularnewline
mv file1 file2 & 移动/重命名文件fil1为file2\tabularnewline
rm file  & 删除文件file\tabularnewline
rmdir directory & 删除目录directory\tabularnewline
cat file & 显示文件file的内容\tabularnewline
more file & 逐屏显示文件file的内容\tabularnewline
head file & 显示文件file的前10行\tabularnewline
tail file & 显示文件file的后10行\tabularnewline
grep 'keyword' file & 在文件file里搜索字符串keyword并显示\tabularnewline
wc file & 分别统计文件file的行数,单词数和字符数\tabularnewline
\hline 
\end{tabular} 


\end{frame}

\subsection{重定向和管道}


\begin{frame}{重定向与管道}

\begin{tabular}{ll}
\hline 
command > file & 将command的输出写入到文件file里\tabularnewline
command >\textcompwordmark{}>file & 将command的输出追加到文件file里\tabularnewline
command <file & 从文件file里读内容当做command的输入\tabularnewline
command1 | command2 & 用管道(pipe)的方式将command1的输出当成command2的输入\tabularnewline
cat file1 file2 >file0 & 将文件file1,file2合成文件file0\tabularnewline
sort & 数据排序\tabularnewline
who & 列出当前登录的用户\tabularnewline
\hline 
\end{tabular}


\end{frame}

\subsection{通配符}


\begin{frame}{通配符}
\begin{columns}%{}


\column{5cm}
\begin{exampleblock}
{*}

匹配任意多个字符\\
\rule[0.5ex]{1\linewidth}{1pt}

\$ ls fglrx{*}

fglrx-amdcccle\_8.632.rpm fglrx-kernel-source.rpm

fglrx-installer\_8.632.changes fglrx-modaliases.rpm
\end{exampleblock}

\column{5cm}
\begin{exampleblock}
? 

匹配任意一个字符

\noindent \rule[0.5ex]{1\linewidth}{1pt}

\$ls list?

list1 list2 list9
\end{exampleblock}
\end{columns}%{}

\end{frame}

\subsection{文件权限}


\begin{frame}[allowframebreaks]{文件权限}

\includegraphics[scale=0.8]{images/basic/fs-permission}


\begin{columns}%{}


\column{5cm}
\begin{exampleblock}
{chmod }

改变文件或者目录的权限

\noindent \rule[0.5ex]{1\linewidth}{1pt}

\% chmod go-rwx biglist 

\% chmod a+rw biglist \end{exampleblock}
\begin{alertblock}
{仅仅文件的所有者或者root账号能改变}
\end{alertblock}

\column{5cm}


\begin{tabular}{cc}
\hline 
u & 文件所有者\tabularnewline
g & 文件属组\tabularnewline
o & 其他账号\tabularnewline
a & 所有账号\tabularnewline
r & 可读\tabularnewline
w & 可写\tabularnewline
x & 可执行(可访问目录)\tabularnewline
+ & 增加权限\tabularnewline
- & 移走权限\tabularnewline
\hline 
\end{tabular}

\end{columns}%{}

\end{frame}

\subsection{获取帮助}


\begin{frame}{获取帮助}
\begin{itemize}
\item 在线帮助

\begin{itemize}
\item man command
\item info command
\item command --help / -h
\item /usr/share/doc/
\end{itemize}
\item 推荐书籍

\begin{itemize}
\item 红旗系统自带印刷手册
\item Linux系统管理技术手册第二版(美)奈米斯 / (美)海因 / (美)斯奈德,张辉译,人民邮电出版社)
\end{itemize}
\item 获取互联网帮助

\begin{itemize}
\item \href{http://www.google.com.hk}{googling}
\item \href{http://www.linuxsir.org}{http://www.linuxsir.org}
\item \href{http://linux.chinaunix.net}{http://linux.chinaunix.net}
\item \href{http://blog.wgzhao.com}{http://blog.wgzhao.com}
\end{itemize}
\end{itemize}

\end{frame}
\end{document}
