\documentclass{beamer}

\usepackage{tikz}
\usepackage{amsmath}
\usetikzlibrary{arrows}
\usepackage{verbatim}
\usepackage{fontspec}
\setsansfont{STXihei}
\begin{document}
\pagestyle{empty}

% For every picture that defines or uses external nodes, you'll have to
% apply the 'remember picture' style. To avoid some typing, we'll apply
% the style to all pictures.
\tikzstyle{every picture}+=[remember picture]

% By default all math in TikZ nodes are set in inline mode. Change this to
% displaystyle so that we don't get small fractions.
\everymath{\displaystyle}




 \tikz\node [fill=blue!20,draw,circle] (n1) {}; 文件的类型 $-$=普通文件 , $d$= 表示目录,$l$=  链接文件 \\
        
  \hspace{5mm} \tikz\node [fill=red!20,draw,circle] (n2) {}; 文件属主权限:r=可读,w=可写,x=可执行,-=无权限 \\
        
 \hspace{10mm} \tikz\node [fill=orange!20,draw,circle] (n3) {}; 成员组权限 \\
            
  \hspace{15mm} \tikz\node [fill=yellow!20,draw,circle] (n4) {};  其他用户权限 \\
     	
  \hspace{20mm} \tikz\node [fill=green!20,draw,circle] (n5) {}; 链接到文件或者目录的个数	    
     	


% Below we mix an ordinary equation with TikZ nodes. Note that we have to
% adjust the baseline of the nodes to get proper alignment with the rest of
% the equation.


\begin{tikzpicture}

%\foreach \x [evaluate=\x as \shade using \x*10] in {1,...,10}
%{
%  \node [ fill=red!\shade!yellow, minimum size=0.65mm] at (\x,0) (t\x) {\x};
%  
%  }

\end{tikzpicture}  

\begin{equation}
        \tikz[baseline]{
            \node [fill=blue!20,anchor=base] (t1)
            {\underline{-}};
        } 
        \tikz[baseline]{
            \node[fill=red!20,anchor=base] (t2)
            {\underline{rw-}};
        } 
        \tikz[baseline]{
            \node[fill=green!20,anchor=base] (t3)
            {\underline{r--}};
        }
        \tikz[baseline]{
        		\node[fill=yellow!20,anchor=base] (t4)
        		{\underline{r-x}};
        	}
        	\tikz[baseline]{
        		\node[fill=green!20,anchor=base] (t5)
        		{\underline{4}};
        	}
        	 \tikz[baseline]{
        		\node[fill=green!20,anchor=base] (t6)
        		{\underline{root}};
        	}
   	    \tikz[baseline]{
        		\node[fill=green!20,anchor=base] (t7)
        		{\underline{wheel}};
        	}      
 	  	\tikz[baseline]{
        		\node[fill=green!20,anchor=base] (t8)
        		{\underline{8237}};
        	}       
   	\tikz[baseline]{
        		\node[fill=green!20,anchor=base] (t9)
        		{\underline{02-23 21:39 2010}};
        	}      	        
   	\tikz[baseline]{
        		\node[fill=green!20,anchor=base] (t10)
        		{\underline{test.txt}};
        	}        
\end{equation}



文件的属主 \tikz\node[fill=purple!20,draw,circle] (n6) {};   \\
文件属于的组 \tikz\node[fill=blue!20,draw,circle] (n7) {};   \\
文件大小 \tikz\node[fill=cyan!20,draw,circle] (n8) {};   \\
日期 \tikz\node[fill=green!40,draw,circle] (n9) {};   \\
文件名 \tikz\node[fill=yellow!40,draw,circle] (n10) {}; 


% Now it's time to draw some edges between the global nodes. Note that we
% have to apply the 'overlay' style.
\begin{tikzpicture}[overlay]
\foreach \x in {1,...,10}
	{
		\path[densely dotted,->] (n\x) edge [bend right] (t\x);
	}
%        \path[->] (n1) edge [bend right] (t1);
%        \path[->] (n2) edge [bend right] (t2);
%        \path[->] (n3) edge [out=0, in=-90] (t3);
%        \path [->] (n4) edge [bend right] (t4);
\end{tikzpicture}

\end{document}