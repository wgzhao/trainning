%linux概述
\section{Linux概览}



\begin{frame}[shrink]{Linux概览}
	\tableofcontents[currentsection]
\end{frame}


\subsection{Linux起源}

\begin{frame}{Linux起源}
\begin{itemize}
\item 1984:GNU项目和自由软件基金(Free Software Foundation,FSF)

\begin{itemize}
\item 创建UNIX工具的开源版本
\item 创建通用公共授权(General Public License,GPL)
\end{itemize}

\item 软件必须开源

\item 1991:Linus Torvalds

\begin{itemize}
\item 创建开源,类UNIX的内核,以GPL方式发布
\item 移植一些GNU工具,网上征求援助
\end{itemize}
\item 今天:

\begin{itemize}
\item Linux kernel + GNU 工具 = 完整的,开源的类UNIX的操作系统
\end{itemize}


\end{itemize}

\end{frame}

\subsection{红旗发行版}


\begin{frame}{红旗(RedFlag)发行版}
\begin{itemize}
\item Linux发行版是基于Linux内核的操作系统
\item 红旗服务器操作系统

\begin{itemize}
\item 稳定,经过完整测试的软件
\item 专业的支持服务
\item 针对大型网络的集中管理工具
\end{itemize}
\item Qomo项目

\begin{itemize}
\item 更多,更新的应该程序
\item 社区支持(非红旗官方支持)
\item 针对个人系统
\end{itemize}
\end{itemize}

\end{frame}

\subsection{Linux原则}


\begin{frame}{Linux原则}


\begin{itemize}
\item 所有对象,包括硬件都是文件
\item 配置数据以文本形式保存 
\item 由短小的单目的程序构成 
\item 避免不必要的用户交互 
\item 可使用多个程序合作完成复杂任务
\end{itemize}

\end{frame}

\section{Linux使用基础}


\begin{frame}{Linux使用基础}
%	\begin{exampleblock}{目标}
%	\begin{itemize}
%		\item 登录系统
%		\item 从终端启动X
%		\item 从X访问命令行
%		\item 修改密码
%		\item 理解root特权的性质
%		\item 提升你的权限
%		\item 编辑纯文本文件
%	\end{itemize}
%	\end{exampleblock}
\center \includegraphics[scale=.6]{images/tux.png}

\end{frame}

\subsection{登录Linux系统}

\begin{frame}{登录系统}
\begin{itemize}
\item 两种登录屏幕类型:虚拟终端(基于文本)和图形化登录(称为显示管理器)
\item 使用登录帐号和密码登录
\item 每一个用户都有自己的主目录用来保存个人文件
\end{itemize}

\end{frame}

\subsection{虚拟终端和图形界面切换}


\begin{frame}{虚拟终端和图形界面切换}
\begin{itemize}
\item 通常Linux运行6个虚拟终端和一个图形终端
\item 在虚拟终端之间切换,按下:\\
Ctrl+Alt+F\emph{{[}1 - 6{]}}
\item 访问图形终端,按下:Ctrl+Alt+F7
\end{itemize}

\end{frame}

\subsection{改变密码}


\begin{frame}{改变密码 }
\begin{itemize}
\item 密码控制访问系统

\begin{itemize}
\item 首次登录时应该修改密码
\item 此后,要经常修改密码
\item 选择难于猜测的字符串作为密码
\end{itemize}
\item 图形下:
\item 终端下:passwd
\end{itemize}

\end{frame}

\subsection{root账户}


\begin{frame}{root账户}
\begin{itemize}
\item root账户:特殊的管理员帐号

\begin{itemize}
\item 也被称为超级用户(superuser)
\item root有着完全控制系统和彻底摧毁系统的能力
\end{itemize}
\item 如无必要,不要使用root登录
\end{itemize}

\end{frame}

\subsection{改变身份}


\begin{frame}{改变身份}
\begin{itemize}
\item su - 为root帐号创建新的shell环境
\item sudo \emph{command} 用root账户运行命令

\begin{itemize}
\item 需要先前配置/etc/sudoers
\end{itemize}
\item id 显示当前用户的信息
\end{itemize}

\end{frame}

\section{运行命令及获取帮助}


\begin{frame}{运行命令和获取帮助}

目标:
\begin{itemize}
\item 在提示符下执行命令
\item 解释某些简单命令的使用帮助摘要
\item 学会如何使用系统内置的帮助资源
\end{itemize}

\end{frame}

\subsection{运行命令}


\begin{frame}{运行命令}
\begin{itemize}
\item 运行的命令一般是下面的语法:

\begin{itemize}
\item command \textit{options} \textit{arguments}
\end{itemize}
\item 每一项用空格隔开
\item Options 可以修改命令的行为

\begin{itemize}
\item 单字母选项一般用-表示,可以每一个选项单独给出,也可以联合
\item -a -b -c or -abc
\item 完整单词的选项一般用--表示
\item -{}-help -{}-usage
\end{itemize}
\item Arguments 可以是命令需要的文件名或者其他数据
\item 一行中的多个命令可以使用 ; 分隔
\end{itemize}

\end{frame}

\subsection{获取帮助}


\begin{frame}{获取帮助}
\begin{itemize}
\item 不必试着记住所有的指令及用法
\item 多层帮助

\begin{itemize}
\item whatis
\item \textit{command} -{}-help
\item man , info
\item /usr/share/doc
\item 红旗文档
\end{itemize}
\end{itemize}
\end{frame}


\begin{frame}{whatis命令}
\begin{itemize}
\item 显示一个命令的简短描述
\item 使用的数据库每日更新
\item 一般情况下,系统刚安装后无效
\item \$ whatis cal\\
cal (1) - displays a calendar and the date of easter
\end{itemize}
\end{frame}


\begin{frame}[fragile]{-{}-help 选项}
\begin{itemize}
\item 显示用法摘要和参数列表
\item 大部分命令都有此参数
\item  
	\begin{verbatim}
	$mount --help
	
	Usage: mount -V : print version
		mount -h : print this help
		mount : list mounted filesystems	
		mount -l : idem, including volume labels
	\end{verbatim}
	\ldots{} \ldots{}

\end{itemize}
\end{frame}

\begin{frame}{阅读用法摘要}
\begin{itemize}
\item 通过-{}-help,man和其他方式打印出用法摘要
\item 用于描述命令的语法
	\begin{itemize}
		\item \emph{[ {} ]} 里的参数表示可选
		\item {<} {>}或者大写的参数表示是变量
		\item 文本后接 \ldots{} 的表示列表重复
		\item $x | y | z$ 表示“x或y或z”
		\item -abc 意思是-a,-b,-c的任意混合
	\end{itemize}
\end{itemize}
\end{frame}
